%% Direttive TeXworks:
% !TeX root = ./ordinamento.tex
% !TEX encoding = UTF-8 Unicode
%% !TEX program = arara
%% !TEX TS-program = arara
% !TeX spellcheck = it-IT

% arara: pdflatex: { synctex: yes, shell: yes }
% arara: pdflatex: { synctex: yes, shell: yes }

%% Genera un file report.xmpdata con i dati di titolo e autore per il formato PDF/A %%
\begin{filecontents*}{\jobname.xmpdata}
  \Title{Algoritmi di ordinamento}
  \Author{Niccolò Maltoni}
  \Copyright{Questo documento è fornito sotto licenza Creative Commons Attribution-ShareAlike 4.0 International}
  \CopyrightURL{http://creativecommons.org/licenses/by-sa/4.0}
\end{filecontents*}

\documentclass[%
  a4paper,            % specifica il formato A4 (default: letter)
  11pt,               % specifica la dimensione del carattere a 12
  oneside,            % serve per impaginare per stampa solo fronte
  notitlepage         % mette il titolo in una pagina separata (solo per article)
]{article}

\usepackage{a4wide}           % consente di avere più spazio nell'A4
\usepackage[T1]{fontenc}      % serve per impostare la codifica di output del font
\usepackage{textcomp}         % serve per fornire supporto ai Text Companion fonts
\usepackage[utf8]{inputenc}   % serve per impostare la codifica di input del font
\usepackage[italian]{babel}   % serve per scrivere Indice, Capitolo, etc in Italiano
\usepackage{lmodern}          % carica una variante Latin Modern prodotto dal GUST
\usepackage{indentfirst}      % serve per avere l'indentazione nel primo paragrafo
\usepackage{setspace}         % serve a fornire comandi di interlinea standard
\usepackage{xcolor}           % serve per la gestione dei colori nel testo
\usepackage{graphicx}         % serve per includere immagini e grafici
\usepackage{float}            % Permette di impostare la posizione assoluta delle immagini
\usepackage{minted} 			    % Permette di inserire codice nel documento

\graphicspath{{./fig/}}

\usepackage[%
  strict,             % rende tutti gli warning degli errori
  autostyle,          % imposta lo stile in base al linguaggio specificato in babel
  italian=guillemets  % imposta lo stile per l'italiano
]{csquotes}                   % serve a impostare lo stile delle virgolette

\usepackage{multirow}         % aggiunge la possibilità di raggruppare celle su più righe nelle tabelle

\onehalfspacing{}             % Imposta interlinea a 1,5 ed equivale a \linespread{1,5}

\setcounter{secnumdepth}{3}   % Numera fino alla sottosezione nel corpo del testo
\setcounter{tocdepth}{3}      % Numera fino alla sotto-sottosezione nell'indice

\usepackage[%
  depth=3,            % equivale a bookmarksdepth di hyperref
  open=false,         % equivale a bookmarksopen di hyperref
  numbered=true       % equivale a bookmarksnumbered di hyperref
]{bookmark}                   % Gestisce i segnalibri meglio di hyperref
\usepackage{hyperref}         % Gestisce tutte le cose ipertestuali del pdf
\hypersetup{%
  pdfpagemode={UseNone},
  hidelinks,          % nasconde i collegamenti (non vengono quadrettati)
  hypertexnames=false,
  linktoc=all,        % inserisce i link nell'indice
  unicode=true,       % only Latin characters in Acrobat’s bookmarks
  pdftoolbar=false,   % show Acrobat’s toolbar?
  pdfmenubar=false,   % show Acrobat’s menu?
  plainpages=false,
  breaklinks,
  pdfstartview={Fit},
  pdfauthor={Niccolò Maltoni},
  pdfcreator={Niccolò Maltoni},
  pdftitle={Algoritmi di ordinamento},
  pdflang={it}
}

\usepackage[a-1b]{pdfx}
\usepackage[%
  italian,            % definizione delle lingue da usare
  nameinlink          % inserisce i link nei riferimenti
]{cleveref}                   % permette di usare riferimenti migliori dei \ref e dei varioref

\title{\LARGE{\textbf{Algoritmi di ordinamento}}}

\author{Niccolò~Maltoni}

\date{}

\begin{document}

  \maketitle
  \clearpage

  \section{Selection sort}\label{sec:selection}

  Il \textbf{Selection sort} (in italiano, \textit{ordinamento per selezione})
  è un algoritmo di ordinamento \textit{in place} che procede selezionando il numero minore nel vettore
  e posizionandolo nella sottosezione ordinata del vettore;
  di fatto suddivide il vettore in due sotsottosezioni:

  \begin{itemize}
    \item una ordinata, che occupa le prime posizioni dell'array;
    \item una non ordinata, che costituisce la parte restante dell'array.
  \end{itemize}

  \noindent{} L'algoritmo può essere considerato la composizione dei problemi di:

  \begin{itemize}
    \item ricerca del minimo in un vettore;
    \item scambio di elementi in un vettore;
    \item scorrimento di un vettore.
  \end{itemize}

  \inputminted{cpp}{./src/selection.cpp}

  \section{Bubblesort}\label{sec:bubble}

  Il \textbf{Bubblesort} (in italiano, \textit{ordinamento a bolle})
  è un algoritmo di ordinamento iterativo \textit{in place} che procede per scambi successivi di elementi adiacenti,
  portando ad ogni iterazione l’elemento massimo nell'ultima posizione del sottovettore.

  La principale differenza rispetto al \textit{Selection sort} sta nel considerare gli elementi a ``bolle''
  (coppie di elementi adiacenti) anziché effettuare una completa ricerca del minimo (o del massimo) ad ogni iterazione.

  \inputminted{cpp}{./src/bubble/simple.cpp}

  \clearpage

  \subsection{Bubblesort con sentinella}\label{subsec:bubble:flag}

  Il \textit{Bubblesort} può essere ottimizzato tramite l'inserimento di una sentinella (\textit{flag}).
  Tramite una variabile booleana, infatti, è possibile verificare se, al termine di un'iterazione,
  sono stati effettuati scambi; qualora non siano avvenuti, non è necessario continuare con le iterazioni.

  \inputminted{cpp}{./src/bubble/flag.cpp}

  \clearpage

  \subsection{Bubblesort ottimizzato}\label{subsec:bubble:opt}

  Il \textit{Bubblesort} (sia base che con sentinella) può essere ottimizzato tenendo conto del fatto che
  dopo ogni iterazione l'elemento più grande si troverà certamente nell'ultima posizione della sottosezione
  non ordinata; di conseguenza, è possibile ridurre di dimensione la sottosezione analizzata ad ogni iterazione.

  \inputminted{cpp}{./src/bubble/optimized.cpp}

\end{document}
